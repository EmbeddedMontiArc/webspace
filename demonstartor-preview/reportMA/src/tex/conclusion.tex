\chapter{Future Work and Conclusion}
This chapter introduces some features which were not implemented, due to the amount of time given for a thesis implementation, or third part applications, that currently do not support required features. Let us consider some possible improvements: \newline
\begin{itemize}
    \item \textbf{Compilation in a web browser:} currently, the toolchain uses the dedicated server for compilation of models. It would be better to simplify the compilation process and carry out it directly on a client side. Clang in browser can not manage the liking of the required mathematical library, details provided in the Architecture chapter.
    \item \textbf{Using different tracks:} in the simulator's core, the integrated track is used, which consists of different kinds of walls, like linear or curved. It would be an improvement to separate the track from the core part, to be able to load different tracks. Even better to have external track constructor, which can build new tracks, like a configuration file for these tracks. In this configuration file, the data related to walls' positions should be stored. Then, it will be possible to construct not only the new tutorials, but new tutorials with various tracks. It increases variability of possible tasks, e.g., we can create a track with some intersections and develop an absolutely new controller. It will be able to manage the crossroad passing.
    \item \textbf{Several controllers in one 3D simulator:} the simulator uses only one controller, which is in charge of the car driving. In the future versions, the simulator could manage several controllers, which being executed simultaneously. It let us to create more complex tutorials with much wider variety of tasks.
    \item \textbf{Standalone tutorials builder:} create tutorials directly in 3D environment, then the tutorial builder will generate the configuration file, which will be used in the simulator. Currently, a configuration file for a tutorial has to be created manually. Moreover, the creation process of new tutorials takes longer, because all data related to objects should be entered manually in a text form. The 3D tutorials' builder could increase the building time significantly.
    \item \textbf{Better objects detection:} currently, the car is uses eight sensors, which measure the distance to objects that located only at the line which begins from the sensor's position and directed strictly like a ray with a given angle. Using this type of measuring, we can not detect types of objects. Moreover, an object can be located just between the measuring sensors and consequently it can not be detected. It would be an significant improvement, if we could detects the shapes of object or some characteristics of objects, which give us an opportunity to determine a type of objects.
\end{itemize}
All offered changes could greatly improve the functionally of the simulator and the education toolchaing in general. A possible disadvantage might be the performance requirements for client side. Currently the car in the simulator runs smoothly even on the PC which has integrated graphical adapter. \newline
\indent In this thesis, the new education environment for learning C\&C modeling language EmbeddedMontiArc has been developed. Moreover, it has already integrated eleven tutorials, which teach student C\&C modeling step-by-step.From the very beginning, we have analyzed the existing tutorials in various fields. The purpose was to find the most important features, which have an influence on the studying process, and even discover the weaknesses and try to overcome them. Then we have figured out the most suitable architecture, which is partially using the already implemented solutions, what decreases the amount of work, and at the same time, does not influence negatively on an user experience. After that, the implementation of missing components in the toolchain was given, which describes the main algorithms and describes which were made during development process. Lately, was shown how to use the tutorials' tool to be productive during development new models. Then, the group of the tutorials was presented. They present: two tutorial for beginners with explanation of basics; two interesting real-world examples, which present the main concepts and reveal important integrated features; one large final tutorial, which summarize the experience derived from all previous tutorials. At the very end, the future possibles enhancements are proposed, which were not implemented due to some technical restrictions, which currently exist, or lack of time, that was given for the thesis implementation.
\cleardoublepage
